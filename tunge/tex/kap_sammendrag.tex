\begin{abstract}
  \noindent P� oppdrag fra SINTEF er det laget en prototype til en
  \emph{tungestyrt musepeker} som kan brukes p� Microsoft Windows via standard
  HID-musegrensesnitt. Musepekeren aktiveres av forspente trykkf�lsomme
  resistanser av typen FSR-400 og FSR-402. Disse sensorene er sv�rt f�lsomme,
  og har en hysteresefunksjon ved kontinuerlig trykk. Det er derfor laget
  en automatisk kalibreringsrutine som bruker av musen kan aktivere manuelt i
  tilfellet musepekeren oppf�rer seg u�nsket.

  For plassering av sensorene er det laget en justerbar \emph{hodeb�yle}.
  Hodeb�ylen er bygd av et noe svakt material, og m�ten sensorene er festet p�,
  b�r forskes videre p�. Ved videreutvikling av den valgte designen kan
  hodeb�ylen bli meget bra.

  Elektronikken som behandler data og <<styrer>> musepekeren, er demokretsen
  \emph{AT90USBKey} fra Atmel. Kretsen har innebygde ADC-er, som benyttes til
  tolking av sensordata. Den gj�r bruk av USB-grensesnittet for overf�ring av
  data, s� vel som energitilf�rsel. Det er integrert en spenningsregulator som
  benyttes til � forspenne sensorene. Regulatoren er sterk nok til � forsyne
  eventuelle tilleggskretser.

  Programkoden er skrevet for � v�re fleksibel ang�ende antall sensorer og
  portene de kobles til, s� vel som valg av mikrokontroller. Programkoden
  foreligger som vedlegg til rapporten.

  Bruk av musen fungerer slik: Brukeren starter en bevegelse og stopper eller
  endrer retning ved �nsket plassering. Pekeren seg langsomt til � begynne med
  for � gi god presisjon. Man kan snakke mens man beveger musepekeren over
  skjermen, og det er mulig � be venstre museknapp v�re aktiv mens man beveger
  musepekeren over skjermen. Det er ogs� mulig � sette musen i en
  <<scroll>>-modus som gj�r surfing av nettsider og lesing av dokumenter meget
  behagelig.

  Utgiftene ved prosjektet var minimale, estimert til ca. 2000 NOK.\\

  \noindent P� bakgrunn av rapporten konkluderer prosjektgruppen med at
  \emph{det er mulig � lage en tungestyrt musepeker med de valgte sensorene}.
  Det mekaniske og ergonomiske m� imidlertid forbedres for � f� et salgbart
  produkt.
\end{abstract}
