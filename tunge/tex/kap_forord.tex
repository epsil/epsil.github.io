\addcontentsline{toc}{chapter}{Forord}
\chapter*{Forord}

Rapporten er en avsluttende bacheloroppgave i elektro- og datateknikk:
fordypning i elektronikk ved H�yskolen i S�r-Tr�ndelag. Oppgaven er definert i
sammarbeid med SINTEF og veileder Herman Ranes fra HiST. Arbeidet er utf�rt i
HiST sine lokaler og har hatt en varighet fra 26. januar 2009 frem til 25. mai
2009. Rapporten er et resultat av dr�ye 4 m�neders forskning og uttesting av en
prototype for for en tungestyrt datamus som kan brukes i Microsoft Windows.
Bakgrunnen for oppgaven er � gi en konklusjon til SINTEF, om det er mulig bruke
trykkresistive f�lere av typen FSR-400 og FSR-402 for dette form�let. Det er
�nsket � kunne gi et svar p� om denne teknologien er tilstrekkelig for � gi en
l�sning p� problemstillingen.

\begin{center}
  \noindent \textbf{\large Vi vil i denne anledning rette en takk til}
\end{center}
\vspace*{-\baselineskip}
\begin{center}
  \emph{Ansatte ved HiST}
\end{center}
\vspace*{-1.5\baselineskip}
\begin{center}
  Herman Ranes

  Rolf Kristian Snilsberg
\end{center}
\vspace*{-\baselineskip}
\begin{center}
  \emph{Personer med tilknytning til rapporten}
\end{center}
\vspace*{-1.5\baselineskip}
\begin{center}
  Tone Berg

  Mats Ekstr�m
\end{center}

\vspace*{5\baselineskip}
Trondheim, 24. mai 2009
